%why qec and principles here, intro to what qc are and what they can do

Quantum computers are devices that exploit the features of quantum mechanics at
the smallest scales of reality. They have the potential to solve computational
problems that cannot be solved in any reasonable time on conventional computers
\cite{fowler12_surfac_codes}. In principle, quantum computers will be able to
accurately simulate physics \cite{feynman82_simul_physic_with_comput}, implement
algorithms, such as Shor's factoring algorithm \cite{Shor_1997}, or Grover's
search algorithm \cite{Grover_1996}, in polynomial time, providing a massive
speedup to computationally hard problems of interest. To date, superconducting
LC circuits have shown the greatest promise in forming effective quantum bits
(or qubits) \cite{Rol_2019} \cite{barends14_super_quant_circuit_at_surfac},
however several other platforms, such as quantum dots
\cite{huang19_fidel_bench_two_qubit_gates_silic} \cite{Lawrie_2020}, NV-centres
in diamond \cite{}, and even topological qubits in semi-conducting nanowires,
are seeing great interest and recent development.


%%% Local Variables:
%%% mode: latex
%%% TeX-master: "QEC_paper"
%%% End:
