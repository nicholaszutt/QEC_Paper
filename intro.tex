Quantum computers are devices that exploit the features of quantum mechanics at
the smallest scales of reality. They even have the potential to solve
computational problems that are not feasible for conventional computers
\cite{nielsen_chuang_2010}. Some of the most famous applications are the
accurate simulation of physics \cite{feynman82_simul_physic_with_comput}, fast
database searching provided by Grover's algorithm \cite{Grover_1996} and the
polynomial time solution for factoring large composite numbers, provided by
Shor's algorithm \cite{Shor_1997}, which has far-reaching consequences for
current methods in cryptography. However, there are still key technological
challenges that need to be overcome before this technology can be realized.

To date, superconducting LC circuits have shown the greatest promise in forming
effective quantum bits (or qubits) \cite{Rol_2019}
\cite{barends14_super_quant_circuit_at_surfac}, however several other platforms,
such as quantum dots \cite{huang19_fidel_bench_two_qubit_gates_silic}
\cite{Lawrie_2020}, NV-centres in diamond \cite{Taminiau_2014}, and even
topological qubits in semi-conducting nanowires \cite{Mourik_2012}, have seen
growing interest and recent development. Nevertheless, they all suffer from some
form of noise, of a slightly different nature in each case, which poses
major challenges in the realization of a scalable quantum computer.

Accounting for this noise is undoubtedly a daunting task. For that, different
approaches have been developed which are usually grouped as part of
\textit{quantum error suppression} (QES), such as dynamical decoupling, which
attempt to reduce the noise at the hardware level, and \textit{quantum error
  correction} (QEC) techniques, which aim to correct errors once they have
occurred. In particular, the latter approach has undergone rapid development in
recent decades from the schemes proposed by Shor \cite{Shor_1995_QEC} and Steane
\cite{Steane_1996_QEC} to the new promising \textit{surface codes}
\cite{fowler12_surfac_codes} that have higher tolerance to errors and require
fewer interactions among the qubits.

In this article, we review the key principles of quantum error correction,
discuss ...



%%% Local Variables:
%%% mode: latex
%%% TeX-master: "QEC_paper"
%%% End:
