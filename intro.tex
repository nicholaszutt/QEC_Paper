Quantum computers are devices that exploit the features of quantum mechanics at
the smallest scales of reality. They even have the potential to solve
computational problems that are not feasible for conventional computers
\cite{nielsen_chuang_2010}. Some of the most famous applications are the
accurate simulation of physics \cite{feynman82_simul_physic_with_comput}, fast
database searching provided by Grover's algorithm \cite{Grover_1996} and the
polynomial time solution for factoring large composite numbers, provided by
Shor's algorithm \cite{Shor_1997}, which has far-reaching consequences for
current methods in cryptography. However, there are still key technological
challenges that need to be overcome before this technology can be realized.

To date, superconducting LC circuits have shown the greatest promise in forming
effective quantum bits (or qubits) \cite{Rol_2019}
\cite{barends14_super_quant_circuit_at_surfac}, however several other platforms,
such as quantum dots \cite{huang19_fidel_bench_two_qubit_gates_silic}
\cite{Lawrie_2020}, NV-centres in diamond \cite{Taminiau_2014}, and even
topological qubits in semi-conducting nanowires \cite{Mourik_2012}, have seen
growing interest and recent development. Nevertheless, they all suffer from some
form of noise, of a slightly different nature in each case, which poses
major challenges in the realization of a scalable quantum computer.

Accounting for this noise is undoubtedly a daunting task. For that, different
approaches have been developed which are usually grouped as part of
\textit{quantum error suppression} (QES), such as dynamical decoupling, which
attempt to reduce the noise at the hardware level, and \textit{quantum error
  correction} (QEC) techniques, which aim to correct errors once they have
occurred. In particular, the latter approach has undergone rapid development in
recent decades from the schemes proposed by Shor \cite{Shor_1995_QEC} and Steane
\cite{Steane_1996_QEC} to the new promising \textit{surface codes}
\cite{fowler12_surfac_codes} that have higher tolerance to errors and require
fewer interactions among the qubits.


\subsection{Surface codes}
\begin{figure}
  \centering
  \includegraphics[width=0.4\textwidth]{images/surface_code.png}
  \caption{A 2D array of qubits representing a surface code of distance 5. Data
    qubits are the white circles, which are connected to 2 $Z$- (green) and 2
    $X$- (yellow) stabilizers respectively. The figure also shows the how to
    define the logical operations $Z_L = Z_6Z_7Z_3Z_8Z_9$ (red) and $X_L =
    X_1X_2X_3X_4X_5$ that anticomute. The purple line stands for an operation on
    the data qubits, $X_s = X_2 X_{10}X_{11} X_{12} $ that does not affect the
    logical state of the qubit (it is a stabilizer). Figure from
    \cite{fowler12_surfac_codes}.}
  \label{fig:surface_code}
\end{figure}
Before discussing the fundamentals of QEC in the following section, we want to
briefly introduce surface codes, and how they can be used to store quantum
states.

%measurement qubits or ancullary qubits?
Surface codes, a class of topological QEC codes, implement error correction
through a 2D array of qubits on a lattice, as shown in Fig.
\ref{fig:surface_code}. Computational states are stored in the \textit{data
  qubits} (open circles), whereas the \textit{ancillary qubits} (solid circles)
perform indirect measurements of the stabilizers. Qubits encoded in the surface
code are protected by stabilizers that only require nearest-neighbour
interactions (making two-qubit gates easy to implement), have a very high error
threshold relative to many other schemes \cite{terhal15}, and are capable of
correcting multiple, arbitrary errors on their physical qubits. Surface codes
are also rather straightforward to build, compared to QEC codes that rely on
more topologically complex architectures (for instance, Kitaev's toric code
\cite{Kitaev_2003}). It is not difficult to see why surface codes have attracted
so much attention recently as the most viable path forward in fault-tolerant
computing.

In this article, we review the key principles of quantum error correction,
discuss the need for physical qubits at error rates below their threshold values
for useful encoding, and briefly present some of the current progress toward
experimental realizations of the surface code on superconducting qubits and
silicon quantum dots.

%%% Local Variables:
%%% mode: latex
%%% TeX-master: "QEC_paper"
%%% End:
