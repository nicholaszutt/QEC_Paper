\documentclass[conference]{IEEEtran}
\usepackage{cite}
\usepackage{amsmath,amssymb,amsfonts}
\usepackage{algorithmic}
\usepackage{graphicx}
\usepackage{textcomp}
\usepackage{xcolor}
% \def\BibTeX{{\rm B\kern-.05em{\sc i\kern-.025em b}\kern-.08em
%     T\kern-.1667em\lower.7ex\hbox{E}\kern-.125emX}}


% Using BibTex, with specified choices of what to include in bibliography
\usepackage[backend=bibtex,
sorting=none, % sort entries by appearance
isbn=true, % include isbn, doi, url etc or not in bibliography entry
doi=false,
url=false,
maxnames=1,
eprint=false,
abbreviate=true,
style=ieee]{biblatex}


% adding the bibliography file
\addbibresource{references}


\begin{document}

\title{Quantum Error Correction: Surface Codes}

\author{
  \IEEEauthorblockN{Asier Galicia}
  \IEEEauthorblockA{\textit{Faculty of Applied Physics} \\
  \textit{Delft University of Technology}\\
    Delft, Netherlands \\
    A.Galiciamartinez@student.tudelft.nl}

  \and

  \IEEEauthorblockN{Nicholas Zutt}
  \IEEEauthorblockA{\textit{Faculty of Applied Physics} \\
  \textit{Delft University of Technology}\\
Delft, Netherlands \\
N.C.F.Zutt@student.tudelft.nl}
}

\maketitle


\begin{abstract}
  This document is a model and instructions for \LaTeX. This and the
  IEEEtran.cls file define the components of your paper [title, text, heads,
  etc.]. *CRITICAL: Do Not Use Symbols, Special Characters, Footnotes, or Math
  in Paper Title or Abstract.
\end{abstract}

\begin{IEEEkeywords}
  quantum computation, error correction, surface code, error decoding, fault-tolerance
\end{IEEEkeywords}

\section{Introduction}
Quantum computers are devices that exploit the features of quantum mechanics at
the smallest scales of reality. They even have the potential to solve
computational problems that are not feasible for conventional computers
\cite{nielsen_chuang_2010}. Some of the most famous applications are the
accurate simulation of physics \cite{feynman82_simul_physic_with_comput}, fast
database searching provided by Grover's algorithm \cite{Grover_1996} and the
polynomial time solution for factoring large composite numbers, provided by
Shor's algorithm \cite{Shor_1997}, which has far-reaching consequences for
current methods in cryptography. However, there are still key technological
challenges that need to be overcome before this technology can be realized.

To date, superconducting LC circuits have shown the greatest promise in forming
effective quantum bits (or qubits) \cite{Rol_2019}
\cite{barends14_super_quant_circuit_at_surfac}, however several other platforms,
such as quantum dots \cite{huang19_fidel_bench_two_qubit_gates_silic}
\cite{Lawrie_2020}, NV-centres in diamond \cite{Taminiau_2014}, and even
topological qubits in semi-conducting nanowires \cite{Mourik_2012}, have seen
growing interest and recent development. Nevertheless, they all suffer from some
form of noise, of a slightly different nature in each case, which poses
major challenges in the realization of a scalable quantum computer.

Accounting for this noise is undoubtedly a daunting task. For that, different
approaches have been developed which are usually grouped as part of
\textit{quantum error suppression} (QES), such as dynamical decoupling, which
attempt to reduce the noise at the hardware level, and \textit{quantum error
  correction} (QEC) techniques, which aim to correct errors once they have
occurred. In particular, the latter approach has undergone rapid development in
recent decades from the schemes proposed by Shor \cite{Shor_1995_QEC} and Steane
\cite{Steane_1996_QEC} to the new promising \textit{surface codes}
\cite{fowler12_surfac_codes} that have higher tolerance to errors and require
fewer interactions among the qubits.

In this article, we review the key principles of quantum error correction,
discuss ...



%%% Local Variables:
%%% mode: latex
%%% TeX-master: "QEC_paper"
%%% End:


\section{Fault Tolerance and Encoding for Error}
#how physical implementations are affected by noise, how to mitigate this
#definition fault tolerance
% I think that here we should mention what the [[n,k,d]] code is.
In the basic theory of quantum error correction, quantum states are encoded into
several physical qubits. By performing the appropriate parity checks, it is
possible to probe the state of the physical qubits without changing the state of
the encoded qubit (also called the \textit{logical qubit}), in order to detect
errors on individual physical qubits. These parity checks are the
\textit{stabilizers} of the error correction code; when all stabilizer
measurements return an eigenvalue of $+1$, it signals that no error has occurred
\cite{nielsen_chuang_2010}. Effective encoding protocols are such that each
error process gives rise to a unique syndrome, which is just the list of
stabilizer measurement outcomes, which provides the information needed to
correct the error \cite{fowler12_surfac_codes}.

In any real implementation of physical qubits, there are several sources of
error. State preparation and measurement may not be carried out perfectly.
Coherent errors also occur through the imperfect application of quantum gates in
circuits \cite{Devitt_2013}. Error models and error correction schemes are all
predicated on the assumption that low-weight errors (affecting only a few
qubits) are more likely than high weight errors. In other words, error
probabilities must be small \cite{terhal15}. In principle though, all these
(small) sources of error can be corrected for using a variety of encoding and
detection schemes.

The error detection scheme implemented by the surface code, shown in Fig.
\ref{fig:surface_code}, is capable of precisely identifying errors in large
arrays of qubits. The $X$- and $Z$-stabilizer ancillas conduct parity
measurements between the data qubits to which they are connected through a
sequence of CNOT operations. Should an error occur on a given data qubit,
depending on the error type ($X$, $Y$, or $Z$) either the $X$ or $Z$ (or both)
type stabilizers will fire (return an outcome of $-1$ in their measurement),
which heralds the error and allows for its appropriate correction. Corrections
often take place at the software level, since applications of gates in the
hardware are susceptible to noise. It is often easier (and less error-prone)to
propagate the error through the circuit and correct the outcome of a measurement
than it is to correct the error in the hardware.

\subsection{Fault tolerance}
As mentioned in the previous section, errors can occur at every point in the
execution of an algorithm. State preparation, measurement, and gate operations
are examples of operations that are affected by these errors. Furthermore, the
errors occurring at any of these steps may propagate through the different
operations to the rest of the components of the system. This could lead to
cascading errors throughout the entire process, destroying all reliability in
the computation through loss of coherence. If a quantum computer is ever to be
realized, errors must be carefully contained.

It is in this context of error propagation that the concept of
\textit{fault-tolerant} quantum computing arises. A certain operation is said to
be fault-tolerant if a single error occurring at any step between two QEC cycles
causes at most one error at each of the logical blocks involved in the operation
\cite{Devitt_2013}. An example of fault-tolerant operation is shown in Fig.
\ref{fig:fault_tol} (b) where a bit-flip error in the top logical block
propagates only once to the bottom logical block (in contrast to (a) where it
propagates multiply). It is noteworthy that the constraint of only one
error propagating to different logical blocks can be relaxed when increasing the
code distance \cite{Devitt_2013}. In fact, for $d$ distance codes, it suffices
to require $d-2$ errors at each logical block.

\begin{figure}[htbp]
  \centering
  \includegraphics[width=0.5\textwidth]{images/fault_tolerance.pdf}
  \caption{Exmple of an operation that is not fault-tolerant (a) and one that is
    fault-tolerant (b). In these examples a bit-flip error in the top logical
    block propagates through the CNOT gates to the bottom logical block.}
  \label{fig:fault_tol}
\end{figure}

\subsection{Error threshold}
One great strength of the surface code is its ability to diagnose multiple
errors simultaneously. Should two adjacent $X$-checks fire at one end of the
surface, while two $Z$-checks fire at the other end, we can conclude that a
phase-flip error occurred on the data qubit between the two $X$-checks, while a
bit-flip error occurred between the $Z$-checks. One must be careful, however,
because decoding the error syndrome for any code is a problem without a unique
solution \cite{terhal15}. Taking the top-left data qubit in Fig.
\ref{fig:surface_code} as an example, one can confirm that a bit-flip error on
that qubit would give rise to the exact same syndrome as a series of bit-flip
errors on the rest of the top row of data qubits. For systems with small error
probabilities, the single bit-flip error is of course much more likely than the
long series of errors required to produce the same syndrome, but as the error
rate grows, such mistakes in identifying the cause of a syndrome become more
commonplace.

When a syndrome is misinterpreted, the code implements the \textit{wrong
  correction}, and this can lead to a logical level operation being applied to
the logical qubit. The rate at which this occurs, called the \textit{logical
  error rate}, as a function of the physical error rate of the code is of great
interest in characterizing the effectiveness of an error correction protocol.

Indeed, we want the logical error rate to decrease as a function of increasing
code distance, either by means of concatenation or, in the case of surface
codes, by increasing the dimensions of surface. However, a high error rate would
prevent this from happening. In the case of surfaces codes, a simple empirical
equation that encapsulates the main properties of the logical error rate
obtained in simulations is \cite{fowler12_surfac_codes}
\begin{equation}
  \label{eq:1}
  P_L = c\left(\frac{p}{p_{th}}\right)^{\frac{ d+1 }{2}},
\end{equation}
where $d$ is the distance of the QEC protocol employed, $c$ is a constant that
depends on the exact characteristics of the error model, $p$ is the physical
error rate under sensible assumptions and $p_{th}$ is the error rate threshold.
This equation clearly shows that the process of encoding is beneficial (leads to
a lower logical error rate) only if the physical qubit error rate is below the
critical threshold value of $p_{th}$. Even though these threshold error rates
vary depending on the employed error model, the current consensus puts the
threshold at around $1\%$ \cite{terhal15} \cite{Versluis_2017}. However,
achieving a physical system that accomplishes such a low error rate
experimentally remains an outstanding challenge.


%%% Local Variables:
%%% mode: latex
%%% TeX-master: "QEC_paper"
%%% End:

\section{Surface Code}
#encoding, detecting errors, ``correcting'' errors in software, fault tolerant
operations?
#decoding algorithms
In order to encode a qubit in a surface code, it suffices to define two logical
operations, namely $Z_L$ and $X_L$, which fulfill the commutation relation
$[Z_L,X_L] = 2Z_LX_L$. When multiple qubits are required, the $X_L$ and $Z_L$
operations among different qubits must commute. Figure \ref{fig:surface_code}
shows one of the possible set of physical operations that can lead to the
required logical operations. In this case, a chain of Z (X) gates in the data
qubits, represented with the red (blue) line, are one of the multiple choices to
define a logical $Z_L$ ($X_L$) operation.

However, even though a set qubits may be fully defined in terms of these
operations, it is still necessary to perform arbitrary rotations and multi-qubit
gates on them. It is a well known result of quantum computation that that it
suffices to be able to apply the standard H, T and CNOT gates to perform
universal computation. However, from Divicenzo criteria
\cite{DiCincenzoCriteria}, we see that an initialization and measurement
capabilities are also necessary.

Although all these operations can be performed using one lattice for each qubit,
is more effective to use a defect based approach. In this case, a qubit is
represented as two Z-cut (X-cut) defects, where, in its simplest form, two Z (X)
stabilizer stops participating in the QEC cycle. Figure \ref{fig:cuts} shows an
example of both Z-cut and X-cut based qubits. The figure also shows with blue
(red) line how to obtain $Z_L$ ($X_L$) operations using $Z$ ($X$) operations in
the data qubits.
\begin{figure}[htbp]
  \centering
  \includegraphics[width=0.5\textwidth]{images/surface_code_cuts.png}
  \caption{Examples of a Z-cut (a) and X-cut (b) qunit in a surface lattice.}
  \label{fig:cuts}
\end{figure}

These logical gates define the qubit but performing operations on them is not
straightforward. Reviewing in full detail all the operations in these defect
based qubits is far beyond the scope of this review, though there are several
techniques that we believe are worth mentioning. The first one is related to the
capability of moving the defects while preserving the state that they represent.
This can be done in several QEC cycles by means of turning off the corresponding
stabilizers and performing several measurements in the necessary data qubits. A
CNOT operation can then be performed moving one of the defects of a Z-Cut qubit
around one of the defects of an X-cut qubit. These operations are called
braiding operations and they are performed ensuring fault tolerance since the
qubits always maintain at least their initial code distance. Furthermore, every
measure explicitly performed on the data qubits can be checked with the
surrounding stabilizers to ensure fault tolerance.

The other technique that worth outlining is the implementation of and arbitrary
Z-rotation (and hence a T gate). This is performed in an indirect way by first
preparing an ancillary. Performing arbitrary rotations require the preparation
of an arbitrary state, which is by itself a challenging task. In this case, the
approach followed consist on starting with a two z-cut separated by only one
data qubit and then separating the z-cut to protect the state from the remaining
of the operation. However the final state will have a low fidelity. That is way
several states are prepared and then a purification procedure is applied, in
such a way that the high number of states are changed for the one desired state
with high fidelity.



%%% Local Variables:
%%% mode: latex
%%% TeX-master: "QEC_paper"
%%% End:


\section{Experimental Implementations}
#surface-7, detection without correction
#surface-17
Implementing a full surface code capable of detecting and correcting a wide
range of possible errors requires at the very least, dozens of physical qubits
capable of performing high fidelity single and two-qubit gate operations. It
isn't surprising then that such a physical implementation remains to date out of
reach. 

Current practical limitations notwithstanding, significant progress has been
made recently in the effort to realize these codes experimentally. Recently,
Andersen \textit{et al.} have realized a experimental setup capable of
repeatedly detecting any single error \cite{Andersen_2020}. Their code is very
small, consisting of just four data qubits and three ancillas (1 $X$-check and 2
$Z$-checks) as shown in Fig. \ref{fig:seven_qbit_code}. Repeated ancilla
measurement was made possible through recent advancements in multi-plexing
readout of superconducting qubits (which were used in this case) which limit
cross-talk and allow for high-fidelity entangling operations and ancilla readout
\cite{barends14_super_quant_circuit_at_surfac} \cite{Bultink_2020}.

Surface codes are capable of successfully detecting $d-1$ errors, and correcting
up to $\lfloor{(d+1)/2} \rfloor$ errors, where $d$ is the code's
\textit{distance}, defined at the smallest number of physical gates required to
perform a logical level operation. Andersen \textit{et al.} define the logical
qubit states
\begin{align}
|0\rangle_L &= \frac{1}{\sqrt{2}} (|0000\rangle + |1111\rangle) , \\
|1\rangle_L &= \frac{1}{\sqrt{2}} (|0101\rangle + |1010\rangle) 
\end{align}
from which it is easy to see that this code has distance $d=2$. It is therefore
capable of \textit{detecting} errors, but not of \textit{correcting} them. This
can be seen by realizing that, for example, an $X$-error on D2 or D4 give rise
to the exact same syndrome.

\begin{figure}
  \centering
  \includegraphics[width=0.4\textwidth]{images/seven_qbit_code.png}
  \caption{A seven qubit surface code, with four red data qubits, one blue
    X-type ancilla and 2 Z-type ancilla qubits. This is the smallest viable
    instance of the surface code.}
  \label{fig:seven_qbit_code}
\end{figure}

Remarkably, despite an inability to explicitly correct for detected errors,
Andersen \textit{et al.}, by measuring ancillas for errors, can ensure a longer
logical qubit lifetime than any physical qubit lifetime for the data qubits
which comprise the logical qubit. By checking repeatedly that no stabilizer
measurement has fired, they show that the decay of the expectation values
$\langle Z_L \rangle$ and $\langle X_L \rangle$ both exceed equivalent
expectation values for the best physical data qubit. Repeated stabilizer
measurement outcomes that signal no error whatsoever are possible here are a
testament to the precision with which superconducting qubits can be initialized,
manipulated and measured. As further research realizes qubits with error rates
further and further below the error threshold for reliable error correction,
larger codes and better stabilizer measurement can make large-scale,
fault-tolerant computation possible. Versluis \textit{et al.} have proposed a
scheme for scalable QEC using transmon superconducting qubits that focuses on
developing an eight-qubit unit cell that can be used to tile together a full
surface code \cite{Versluis_2017}.

Implementations of the surface code are not only limited to superconducting
qubits, either. Recent proposals, such as that of Hill \textit{et al.} , to
implement the surface code in silicon quantum dots promise to drastically reduce
the overhead per qubit by exploiting the uniformity between phosphorus donor
nuclear spin qubits \cite{silicon_surface_code}. Their proposal addresses the
requirement for repeated stabilizer measurement by implementing shared control
over all the qubits, resulting in a control architecture capable of executing
the full sequence of CNOT gates for all ancilla qubits in just four steps. The
number of steps required is independent of qubit number, which makes this
approach eminently scalable. Though the results presented by Hill \textit{et
  al.} are simulations, the authors stress that all the key requirements for
implementing this scheme in a physical system have already been demonstrated.

%%% Local Variables:
%%% mode: latex
%%% TeX-master: "QEC_paper"
%%% End:


\section{Conclusion}
Quantum error correction, and its realization experimentally, remains a key
milestone in the development of a scalable, fault-tolerant quantum computer.
Progress in the development of high threshold error correcting codes, especially
the surface code, along with the fault-tolerant application of quantum gates at
the logical level, provide a possible road-map to a truly universal quantum
computer.

Many challenges have yet to be addressed, though. Higher fidelity control and
measurement of qubits would certainly help in creating codes with physical error
rates below threshold, but scaling up to system sizes beyond that of just a
handful of qubits also necessitates the development of novel electrical control
architectures capable of issuing extremely precise manipulation and read-out
instructions to a large number of qubits in parallel, while likely operating at
cryogenic temperatures.

A surface code comprised of 9 data qubits and 8 ancilla qubits, commonly called
the surface-17 code, is the smallest possible surface code capable of detecting
and correcting arbitrary single-qubit errors
\cite{obrien17_densit_matrix_simul_small_surfac}. The experimental
implementation of such a surface-17 code that shows an increase in the fidelity
of the logical qubit, would constitute a great step forward in the realization
of fault-tolerant QEC.


%%% Local Variables:
%%% mode: latex
%%% TeX-master: "QEC_paper"
%%% End:


\printbibliography
\end{document}
