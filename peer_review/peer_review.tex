% Created 2020-07-03 Fri 11:38
% Intended LaTeX compiler: pdflatex
\documentclass[11pt]{article}
\usepackage[utf8]{inputenc}
\usepackage[T1]{fontenc}
\usepackage{graphicx}
\usepackage{grffile}
\usepackage{longtable}
\usepackage{wrapfig}
\usepackage{rotating}
\usepackage[normalem]{ulem}
\usepackage{amsmath}
\usepackage{textcomp}
\usepackage{amssymb}
\usepackage{capt-of}
\usepackage{hyperref}
\author{Asier Galicia, Nicholas Zutt}
\date{03.07.2020}
\title{Peer Review of Group 14 by Group 8}
\hypersetup{
 pdfauthor={Asier Galicia, Nicholas Zutt},
 pdftitle={Peer Review of Group 14 by Group 8},
 pdfkeywords={},
 pdfsubject={},
 pdfcreator={Emacs 26.3 (Org mode 9.3.6)}, 
 pdflang={English}}
\begin{document}

\maketitle

\section{Summary of the paper}
\label{sec:org1920089}

Microwave drivers are key electronic components of the architectures
that control experimental quantum processors. As quantum computers
scale up to sizes necessary for fault-tolerance, the use of
conventional room-temperature controls face new challenges. The paper
reviews a recent approach to scaling up control using microwave
drivers at cryogenic temperatures. This lowers delay times by bringing
control architectures closer to the quantum processor. Further, they
discuss the use of frequency-division multiple access control, which
allows parallel, multiple access to qubits such that the number of
control lines grows slower than the number of qubits. The paper
concludes with two state-of-the-art examples of such cryogenic
microwave driver technology, developed by Google and Intel.

\section{Quality of the content}
\label{sec:org0be0e18}

\subsection{Have all points been addressed?}
\label{sec:org8901d4b}

\subsubsection{Are microwave drivers required for solid-state qubits? (5/5)}
\label{sec:org7b3309a}
They mention three platforms, trasmon qubits, single electron
semiconductor platforms and NV centres in diamond where microwave
drivers are the go-to technology to perform single qubit
rotations. The authors also mention a key disadvantage of using
microwave drivers, which is the inherent trade-off between power
consumption and gate speed, which becomes a greater challenge at
cryogenic temperatures. However, they also mention a platform
based on multiple electron spin qubits that can implement single
qubit rotation by means of the Heisenberg exchange operation,
raising the possibility that microwave drivers are not essential
for qubit control in all hardware platforms.

Given that they give a complete, clear description of the use of
microwaves in several platforms, along with advantages and
disadvantages, we believe the authors deserve 5 points in this
question.

\subsubsection{Room temperature vs cryogenic MW drivers: pros and cons. (4/5)}
\label{sec:org5e1809f}
The authors aptly outline the cons associated with room
temperature microwave drivers, namely the delay (of up to 10ns)
and thermal load caused by long cables running from the drivers to
the processor. However, they do not explicitly mention advantages
of room temperature drivers, which could be for example their
simplicity or, as mentioned in Bardin \emph{et al.}, the fact that power
consumption is not an issue at room temperature and one can use
well established technology.

Regarding the cryogenic drivers, the authors explain how low
temperature drivers can address the main disadvantages associated
with room temperature drivers. They also outline the challenges
associated with implementing drivers at cryogenic temperatures.

The authors did not specifically address the pros of room
temperature drivers, but otherwise adequately explained the
trade-offs between room temperature and cryogenic drivers. For
this reason, we believe they deserve 4 points for this section.

\subsubsection{Advantages and requirements of FDMA. (5/5)}
\label{sec:orgb526cf4}
The authors give a good overview of FDMA and explain how it can
take advantage of the tunability of qubit frequencies in order to
address many qubits through one microwave line, while allowing for
parallel operations (in contrast to time-division multiple access
approaches). 

They also explain that implementing FDMA requires drivers with a
larger range of frequencies available to them, as well as more
complex read-out components. One final cost of FDMA is introducing
potential cross-talk between qubits, something that will require
more complex electronics to mitigate and control.

Since they address explicitly both advantages and requirements of FDMA and
they also give complementary information regarding to the constrains related
to FDMA, we believe the authors deserve 5 points.

\subsubsection{Implications of cryogenic electronics on MW driver design. (4/5)}
\label{sec:org9dd003e}
The authors discuss how moving to cryogenic temperatures will
impact the design of microwave drivers pointing out the need
to "balance power consumption, performance, and robust component
selection". Power consumption and performance seem to be addressed
adequately, using the Google controller and Horse Ridge as
examples from the state of the art. However, more details about
how robust component selection affects the design choices could
have been included.

As a result, we believe the authors deserve 4 points for this section.

\subsection{Other important points that have been treated}
\label{sec:org90c8a12}
The authors include a helpful discussion about implementing universal
gate sets on the various hardware platforms mentioned in their paper.

They also discuss why scalable electronics are necessary for the
implementation of a fully fault-tolerant quantum computer and mention
the threat of leakage outside the computational subspace, especially
for transmon qubits.

Finally, they include the NV-centres as part of the solid state platforms that
make use of microwave drivers. This is not addressed in the starting
bibliography but is still relevant to the presented discussion.

\subsection{Analysis of state of the art}
\label{sec:org43b4637}
The main references provided for this paper were by Bardin \emph{et al.} and
Patra \emph{et al.} which we employed by the authors several times. The
authors provided a clear overview of the results and methods that
Bardin \emph{et al.} used to develop their cryogenic driver, discussing the
advantages and limitations at length. They also use the paper as a
reference for discussing the power consumption limits of cryogenic
microwave drivers, specifically mentioning the 250 $ \mu $W per qubit limit.

The second main reference by Patra \emph{et al.} was employed in the
state-of-the-art section to illustrate the types of design choices made when
working with microwave drivers at cryogenic temperatures, in this case improving
upon the results of Bardin \emph{et al.} by including FDMA capabilities and
extending the range of available frequencies to allow for both transmon and spin
qubit control. The review was concise and showed clear understanding of the
results presented by Patra \emph{et al.}.

There is another paper that has been extensively employed as reference material
which is the reference [5] by Jeroen P. G. \emph{et al.} This reference appears
several times all over the paper to support valuable data that helps in the
understanding of the topic. Examples of that are the 10ns delay due to cable
length that is comparable to the gate time. These gate times are most likely
obtained from table 1 and the discussion of section 4 of that same reference.


Finally, there are several other references included to support specific
statements that show a wide knowledge of the topic and give strength and
veracity to the text. "It make the paper look professional."


\section{Clarity of the paper}
\label{sec:orgd9b6e0d}

\subsection{Structure}
\label{sec:org38a86f9}
The paper was properly structured and followed the order outlined in
the introduction, making it very easy to read. One suggestion would be
to have made the "Multiplexing Techniques" and "Microwave Drivers at
Cryogenic Temperatures" sections into sub-sections under a section
head entitled "Proposed Solutions". This would have made the structure
clearer and would have coincided better with the paper overview given
at the end of the introduction.

\subsection{Language}
\label{sec:orge30959e}
The language was very clear. A writing style of short, declarative sentences
broke down larger concepts effectively. The paper was grammatically correct
throughout with hardly any typos or other mistakes, the most noticeable one
being in section 5 were they say that the phase needs to be kept within a 0.22\%
tolerance but it should be (Berdin \emph{et al.}) $0.22^\circ$. Nevertheless, it
was well-written and clearly proof-read several times. 

\subsection{Formatting}
\label{sec:org94cb047}
There were no figures and the one equation in the paper was
well formatted with well defined symbols. The bibliography is well
formatted and complete with helpful links for references throughout
the paper and DOI links in the reference list. However the
bibliography does not appear in any particular order, contrary to
common approaches that organize entries by appearance in the text, or
name, or date.

\section{Additional remarks}
\label{sec:org6656e70}
The authors wrote in an engaging way, illustrating the implications of cryogenic
temp on driver design using the current state of the art. Accomplished a lot in
one section.

In Section 5, the authors write that Bardin \emph{et al.} use 0.01\% as a
"benchmark" of the total error rate, which rendered the rest of the
paragraph unclear. We thought that their use of the word benchmark
meant that Bardin \emph{et al.} were using an overall error rate of 0.01\% as
the standard against which to compare their results. But, after
reviewing the paper, we suspect that they mean that Bardin \emph{et al.}
provided specifications corresponding to an overall rate of 0.01\%.
This point could have been communicated more clearly. 

There were sections of the paper where a figure could have
added greatly to explanations. We understand that the authors likely
chose not to include any figures due to the page limit, but we would
argue that figures can be a better and more efficient way of
explaining a concept or experimental setup than words alone.

Overall, the paper was clearly written, well-researched and showed
good understanding of the concepts and challenges in the development
of cryogenic microwave drivers.
\end{document}