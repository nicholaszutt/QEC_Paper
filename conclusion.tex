Quantum error correction, and its realization experimentally, remains a key
milestone in the development of a scalable, fault-tolerant quantum computer.
Progress in the development of high threshold error correcting codes, especially
the surface code, along with the fault-tolerant application of quantum gates at
the logical level, provide a possible road-map to a truly universal quantum
computer.

Many challenges have yet to be addressed, though. Higher fidelity control and
measurement of qubits would certainly help in creating codes with physical error
rates below threshold, but scaling up to system sizes beyond that of just a
handful of qubits also necessitates the development of novel electrical control
architectures capable of issuing extremely precise manipulation and read-out
instructions to a large number of qubits in parallel, while likely operating at
cryogenic temperatures.

A surface code comprised of 9 data qubits and 8 ancilla qubits, commonly called
the surface-17 code, is the smallest possible surface code capable of detecting
and correcting arbitrary single-qubit errors
\cite{obrien17_densit_matrix_simul_small_surfac}. The experimental
implementation of such a surface-17 code that shows an increase in the fidelity
of the logical qubit, would constitute a great step forward in the realization
of fault-tolerant QEC.


%%% Local Variables:
%%% mode: latex
%%% TeX-master: "QEC_paper"
%%% End:
