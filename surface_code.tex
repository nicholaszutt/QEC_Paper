The discussion thus far has focused mainly on encoding one logical state in the
entire 2D array of qubits that make up a particular surface. However, creating
different surfaces for each logical qubit is not the most effective way to
scale up this scheme for multiple logicals. One key candidate approach suggests
encoding qubits using \textit{defects} in the lattice structure of the surface
code, by switching off certain parity measurements to define logical qubits. We
will now outline this procedure.

In order to encode a qubit in a surface code, it suffices to define two logical
operations, namely $Z_L$ and $X_L$, which fulfill the commutation relation
$[Z_L,X_L] = 2Z_LX_L$. When multiple qubits are required, the $X_L$ and $Z_L$
operations among different qubits must commute. Figure \ref{fig:surface_code}
shows one of the possible set of physical operations that can lead to the
required logical operations. In this case, a chain of Z (X) gates in the data
qubits, represented with the red (blue) line, are one of the multiple choices to
define a logical $Z_L$ ($X_L$) operation.

\begin{figure}[htbp]
  \centering
  \includegraphics[width=0.5\textwidth]{images/surface_code_cuts.pdf}
  \caption{Examples of a Z-cut (a) and X-cut (b) qubit in a surface code
    lattice. The logical gates are defined such that they satisfy the
    appropriate commutation relations. Moving defects further from each other
    can increase the code distance (the minimum number of gates required for a
    logical operation). Figure from \cite{fowler12_surfac_codes}.}
  \label{fig:cuts}
\end{figure}

However, even though a set qubits may be fully defined in terms of these
operations, it is still necessary to perform arbitrary rotations and multi-qubit
gates on them. It is a well-known result of quantum computation that that it
suffices to be able to apply the standard H, T and CNOT gates to perform
universal computation \cite{nielsen_chuang_2010}. However, from the DiVincenzo
criteria \cite{DiCincenzoCriteria}, we see that initialization and
measurement capabilities are also necessary.

These operations can be performed using one lattice per qubit, but a defect
based approach is much more efficient. In this case, a qubit consists of two
Z-cut (X-cut) defects, where, in its simplest form, two Z (X) stabilizers stop
participating in the QEC cycle. Figure \ref{fig:cuts} shows an example of both
Z-cut and X-cut qubits. The figure also shows with blue (red) line how to obtain
the $Z_L$ ($X_L$) operation using $Z$ ($X$) operations in the data qubits.

These logical gates define the qubit but performing arbitrary operations on them
is highly non-trivial. Reviewing in full detail all the operations in these
defect based qubits is beyond the current scope, though there are several
techniques that are worth mentioning. The defects that make up the qubits on the
plane can be moved while preserving their logical states, and a CNOT operation
can be performed by moving one of the defects of a Z-cut qubit around one of the
defects of an X-cut qubit. These \textit{braiding operations} are performed
ensuring fault tolerance since the qubits always maintain at least their initial
code distance. Furthermore, every measure explicitly performed on the data
qubits while moving the defects, can be checked with the surrounding stabilizers
to ensure fault tolerance.

The other technique worth outlining is the implementation of arbitrary
Z-rotations (making a T gate possible). This is performed in an indirect way by
first preparing an ancilla. Performing arbitrary rotations require the
preparation of an arbitrary state, which is by itself a challenging task. In
this case, the approach followed consists of starting with a two Z-cut qubits
separated by only one data qubit, where the desired state is easily obtained by
proper unitary gates. Afterwards, the Z-cuts are separated to protect the state
for the remainder of the operation. Due to the expected low fidelity of the
prepared ancilla, several such ancillas would be prepared and would undergo
purification to create one high fidelity ancilla for use. These techniques make
clear how a universal gate set can be implemented using the surface code.


%%% Local Variables:
%%% mode: latex
%%% TeX-master: "QEC_paper"
%%% End:
